\documentclass[a4paper,spanish,10pt]{article}
\usepackage{amssymb,amsmath,amsfonts}
\usepackage[utf8]{inputenc}
\usepackage{anysize}
\usepackage[all]{xy}
\usepackage{multicol}
\usepackage{listings}
\usepackage[pdftex]{graphicx}
\marginsize{1cm}{1cm}{1cm}{1cm}
\title{Material de referencia\\Astrum}
\date{}
\begin{document}
	\maketitle
	\begin{center}
		\includegraphics[scale=0.25]{cucei.jpg}
		\\
		\includegraphics{acm.png} 
	\end{center}
	\setcounter{page}{0}	
	\newpage
	Programaci\'on din\'amica\\
	\begin{itemize}
		\item Eliminaci\'on de la recurci\'on. 
		\item {
			LCS(X,Y), Longes Common Subsequence entre 2 cadenas. \\
			$S_{i,j}=
			\begin{cases}
				0&i,j=0\\
				max(S_{i-1,j-1}+1,S_{i-1,j},S_{i,j-1})&X_i=Y_j\\
				max(S_{i-1,j},S_{i,j-1})
			\end{cases}\\
			$
		}			
	\end{itemize}
					\begin{lstlisting}[language=C, numbers=left,tabsize=4]
for(i = 0; i <= n; i++) D[i][0] = 0;
for(j = 0; j <= m; j++) D[0][j] = 0;
for(i = 1; i <= n; i++) 
	for(j = 1; j <= m; j++)
		if(x[i] == y[j]) 
			D[i][j] = max(D[i-1][j], D[i][j-1], D[i-1][j-1] + 1);
		else 
			D[i][j] = max(D[i-1][j], D[i][j-1]);
	\end{lstlisting}
	\begin{itemize}
		\item LIS, Longest Increasing subsequence. Secuencia $S$ que cumple que $\forall(i,j)\{0 \le i \le j < |S| , S_i \le S_j\}$ 
		\item $LIS(i)=MAX(1,LIS(j)+1 ), j < i, S_j \le S_i$
		\item {
			DP de la mochila. 
			\begin{itemize}
				\item $C$ = Costos
				\item $B$ = Beneficios
				\item $P$ = Presupuesto
				\item $S(i,j)=  
				\begin{cases}
					max(S(i-1,j),S(i-1,j-C_i)+B_i) & j-C_i \ge 0\\
					S_{i-1,j}
				\end{cases}
				$
			\end{itemize}
		}
	\end{itemize}
	\begin{lstlisting}[language=C,numbers=left,tabsize=4]
for(int i=1;i<=Proyectos;i++)
	for(int j=1;j<=Presupuesto;j++)
		if(j-C[i]>=0)
			S[i][j]=max(S[i-1][j],S[i-1][j-C[i]]+B[i]);
		else
			S[i][j]=S[i-1][j];
	\end{lstlisting}
	\setcounter{page}{14}	
	\newpage
	Teor\'ia de n\'umeros\\
	\begin{itemize}
		\item{
			Primalidad\\
			\begin{equation}
				p \in \mathbb{P} \leftrightarrow (\forall  x\in [2,\sqrt{p}] , x \nmid p)
			\end{equation}
			Criba de Erat\'ostenes\\
			\begin{enumerate}
				\item$esPrimo(p) = verdadero, \forall p \in \mathbb{N} $
				\item$esPrimo(1) = falso $
				\item$\forall i esPrimo(i)[ esPrimo(j)=falso, \forall j (i\mid j) ]$
			\end{enumerate}
		}
		\item{
			Factores primos $p$ en $n!$\\
			\begin{equation}
				\Bigg\lfloor\frac{n}{p}\Bigg\rfloor+\Bigg\lfloor\frac{n}{p^2}\Bigg\rfloor+\Bigg\lfloor\frac{n}{p^3}\Bigg\rfloor+...=\sum_{i=1}^\infty {\Bigg\lfloor \frac{n}{p^i} \Bigg\rfloor} 
			\end{equation}
		}
		\item{
			M\'aximo Com\'un Divisor
			\begin{equation}
				MCD(a,b)=
				\begin{cases}
					a&b=0\\
					&MCD(b,a \mod b)
				\end{cases}
			\end{equation}
		}
		\item{
			M\'inimo Com\'un M\'ultiplo
			\begin{equation}
				MCM(a,b)=\frac{ab}{MCD(a,b)}
			\end{equation}
		}
		\item {
			\begin{itemize}
				\item {$\varphi(m) = |\{n \in \mathbb{N} | n \leq m \land mcd(m, n) = 1 \}| $
				
					\begin{equation}
						\varphi(n)=n\prod_{ \forall p \in \mathbb{P} , p \mid n}{\left(1-\frac{1}{p}\right)}
					\end{equation}
					
				\item $p,q \in \mathbb{P}, \varphi(pq)=\varphi(p)\times\varphi(q)$
				\item $p \in \mathbb{P}, \varphi(p^k)=(p-1)p^{k-1}$
				}
			\end{itemize}
		}
		\item{
			Aritm\'etica modular
			\begin{itemize}
				\item $a \equiv b \pmod{n}$ 
				\item $-a \equiv n-a \pmod{n}$
				\item si $MCD(a,n)=1$, entonces $\exists a^{-1}(aa^{-1}\equiv 1 \pmod{n})$
				\item $a^{-1}\equiv a^{\varphi(n)} \pmod{n}$
				\item{
					Algoritmo de Euclides Extendido
					\begin{lstlisting}[language=python, numbers=left,tabsize=4]
Euclides_Extendido(a,b):
	if b == 0:
		return  a,1,0
	d,x,y = Euclides_Extendido (b,a % b)
	return d,y,x-(a/b)*y

inverse(a,p):
	d,x,y=Euclides_Extendido(a,p)
	return (x+p)%p
					\end{lstlisting}
				}
			\end{itemize}
		}
	\end{itemize}
	\newpage
	Combinatoria y Distribuciones de Probabilidad\\
	\begin{itemize}
		\item $\binom{n}{k} =\frac{n!}{k!(n-k)!}$, Formas de tomar $k$ elementos de entre $n$
		\item $\frac{n!}{(n-k)!}$, Formas de ordenar $k$ elementos de entre $n$
		\item $\binom{n+k-1}{k-1}$, Formas de sumar $n$ con $k$ sumandos
		\item Números de Catalan = $\frac{(2n)!}{(n+1)!n!} \\
			C_0 = 1, C_{n+1} = \frac{2(2n+1)}{n+2}C_n$
		\item Probabilidad = $\frac{\text{Casos Favorables}}{\text{Casos Totales}}$
		\item Eventos mutuamente incluyentes $P(A \cup B)=P(A)+P(B)-P(A\cap B)$
		\item Eventos mutuamente excluyentes $P(A \cup B)=P(A)+P(B)$
		\item Regla de Bayes $P(A|B)=\frac{P(A\cap B) }{P(A)}$
		\item {
			Distribuci\'on Binomial
			\begin{itemize}
				\item[$\star$] Experimentos de Bernoulli con remplazo. 
				\item[$\star$] $p=\text{Probabilidad de \'exito}$
				\item[$\star$] $q=1-p=\text{Probabilidad de fracaso}$
				\item[$\star$] $x= \text{cantidad de \'exitos}$
				\item[$\star$] $n=\text{Cantidad de experimentos}$
				\item[$\star$] $P(x)=\binom{n}{x}p^xq^{n-x}$.   
			\end{itemize}
		}
		\item{
			Distribuci\'on Hipergeom\'etrica
			\begin{itemize}
				\item[$\star$] Experimentos de Bernoulli sin remplazo. 
				\item[$\star$] $N=\text{Tama\~no de la poblaci\'on}$
				\item[$\star$] $A=\text{Cantidad de elementos exitosos en la poblaci\'on}$
				\item[$\star$] $n=\text{Tama\~no de la muestra(experimentos)}$
				\item[$\star$] $X=\text{N\'umero de \'exitos en la muestra}$
				\item[$\star$] $P(x)=\frac{\binom{A}{x}\binom{N-A}{n-x}}{\binom{N}{n}}$
			\end{itemize}
		}
		\item{
			
			Distribuci\'on Uniforme
			\begin{itemize}
				\item [$\star$] $P(x)=\frac{1}{n}$
				\item [$\star$]  Misma probabilidad para los $n$ eventos.
			\end{itemize}
		}
		Tri\'angulo de Pascal\\
		\xymatrix{
			&&&&\binom{0}{0}=1\ar@{->}[dl]|+\ar@{->}[dr]|+&&&&\\
			&&&\binom{1}{0}=1\ar@{->}[dl]|+\ar@{->}[dr]|+&&\binom{1}{1}=1\ar@{->}[dl]|+\ar@{->}[dr]|+&&&\\
			&&\binom{2}{0}=1\ar@{->}[dl]|+\ar@{->}[dr]|+&&\binom{2}{1}=2\ar@{->}[dl]|+\ar@{->}[dr]|+&&\binom{2}{2}=1\ar@{->}[dl]|+\ar@{->}[dr]|+&&\\
			&\binom{3}{0}=1\ar@{->}[dl]|+\ar@{->}[dr]|+&&\binom{3}{1}=3\ar@{->}[dl]|+\ar@{->}[dr]|+&&\binom{3}{2}=3\ar@{->}[dl]|+\ar@{->}[dr]|+&&\binom{3}{3}=1\ar@{->}[dl]|+\ar@{->}[dr]|+&\\
			\binom{4}{0}=1&&\binom{4}{1}=4&&\binom{4}{2}=6&&\binom{4}{3}=4&&\binom{4}{4}=1
		}
	\end{itemize}
	\newpage
	STL: Standar Template Library
	\begin{itemize}
		\item{ algorithm
			\begin{itemize}
				\item for\_each(ini, fin). Punteros al inicio y fin de la estrctura
				\item sort(ini, fin [,com()]). Punteros al inicio y fin del arreglo, funcion de comparacion.
				\item stable\_sort(ini, fin [,com()]). Punteros al inicio y fin del arreglo, funcion de comparacion.
				\item binary\_search(ini, fin, val, [comp()]). Punteros al inicio y fin del arreglo, busca a val, funcion de comparacion, como en sort. 
				\item nth\_element(ini, nth, fin[comp()]). Ordena parcialmente los datos de modo que el enesimo llegue a su posicion. 
				\item min\_element(ini, fin, [com()]). Retorna puntero al dato menor en el arreglo.
				\item max\_element(ini, fin, [com()]). Retorna puntero al dato mayor en el arreglo.
				\item next\_permutation(ini, fin, [comp()]). Genera la siguiente permutacion del arreglo. Retorna falso si no fue posible y quedaron ordenados de menor a mayor
				\item prev\_permutation(ini, fin, [comp()]). Genera la anterior permutacion del arreglo. Retorna falso si no fue posible y quedaron ordenados de mayor a menor
			\end{itemize}
		}
	\end{itemize}
	\begin{multicols}{2}
	\begin{itemize}
		\item{ queue
			\begin{itemize}
				\item empty() retorna verdadero si esta vacio
				\item size() retorna la cantidad de elementos
				\item push() inserta dato al final
				\item pop() elimina el dato siguiente
				\item back() regresa el ultimo dato
				\item front() regresa el dato siguiente
			\end{itemize}
		}
		\item{ stack
			\begin{itemize}
				\item empty() retorna verdadero si esta vacio
				\item size() retorna la cantidad de elementos
				\item push() inserta dato al final
				\item pop() elimina el dato siguiente
				\item top() regresa el dato siguiente
			\end{itemize}
		}
		\item{ priority\_queue
			\begin{itemize}
				\item empty() retorna verdadero si esta vacio
				\item size() retorna la cantidad de elementos
				\item push(elem) inserta dato al final
				\item pop() elimina el dato siguiente
				\item top() regresa el dato siguiente
				\item *Sobrecargar el operador $<$ si se usan clases. 
			\end{itemize}
		}
		\item{ map
			\begin{itemize}
				\item iteradores: begin(), end(), find(key). find retorna end() si no es encontrado
				\item count(key) retorna la cantidad de elementos con la llave a buscar. 
				\item erase(key) elimina datos con la llave deseada
				\item clear() reinicia la estructura
				\item operator[] retorna por referencia el elemento con la llave deseada
				\item empty() retorna verdadero si esta vacio
				\item size() retorna la cantidad de datos en el mapa
			\end{itemize}
		}
		\item{ vector
			\begin{itemize}
				\item begin() apuntador al inicio
				\item end() apuntado al final
				\item size() retorna la cantidad de elementos
				\item max\_size() retorna la cantidad maxima de elementos
				\item empty() retorna verdadero si esta vacio
				\item operator[] retorna el dato en la posicion deseada
				\item at() retorna el dato en la posicion deseada
				\item back() retorna el ultimo elemento
				\item front() retorna el primer elemento
				\item push\_back() inserta al final
				\item pop\_back() elimina el ultimo elemento
				\item insert() posicion,valor
				\item erase() posicion | inicio, fin (apuntadores)
				\item clear() reinicia vector
			\end{itemize}
		}
	\end{itemize}
	\end{multicols}
	\newpage
		Geometr\'ia
		\begin{itemize}
			\item{
				$CCW(A,B,C)=(B-A) \times (C-A) $
			}
			\item{
				Producto cruz
				\begin{itemize}
					\item $u=(a,b)$
					\item $v=(c,d)$
					\item $u \times v = (ad-bc)\hat k $
				\end{itemize}
			}
			\item{
				Convex Hull (Graham Scan)
				\begin{enumerate}
					\item Punto origen = Punto mas a la izquierda
					\item Inicializar Pila S
					\item For i=0...n:
						\begin{itemize}
							\item A=tope de S, B=siguiente de A en S y C=i\'esimo punto
							\item si CCW(B,A,C)$<0$, sacar de la pila y repetir desde el paso anterior.
							\item Agregar C a S
						\end{itemize}
					\item Puntos en S son el Convex Hull
				\end{enumerate}
			}
		\end{itemize}

\begin{lstlisting}[language=C++, numbers=left,tabsize=4]

bool operator<(vec a,vec b)
	return atan2(a.x,ay)<atan2(b.x,b.y)

double cross(vec a, vec b)
	return a.x*b.y-a.y*b.x

double ccw(vec a, vec b,vec c)
	return cross((b-a),(c-a))

bool choque(vec a, vec b, vec c, vec d)
	return ccw(a,b,c)*ccw(a,b,d)<0 && ccw(c,d,a)*ccw(c,d,b)<0;

bool orden(vec a, vec b)
	a=a-ini;
	b=b-ini;
	return a<b;

void convexHull()
	sort(puntos,puntos+n,orden)
	puntos[n]=puntos[0]
	int i=2
	S.push_back(puntos[0])
	S.push_back(puntos[1])
	while(i<=n)
		int s=S.size()
		if(s>=2 && ccw(S[s-2],S[s-1],puntos[i])>=0)
			S.pop_back()
		else
			S.push_back(puntos[i++])
\end{lstlisting}
			
\newpage
Ley del seno\\
$\frac{a}{\sin \alpha } = \frac{b}{\sin \beta} = \frac{c}{\sin \gamma} $\\
Ley del coseno\\
$c^2 = a^2+b^2-2ab\cos{\gamma}$\\
Area de un triangulo \\
$A = \sqrt{s(s-a)(s-b)(s-c)}$, donde $s = \frac{a+b+c}{2}$\\
Series de Taylor\\
$f(x) \approx \sum_{n=0}^\infty\frac{f^{(n)}(a)}{n!}(x-a)^n$ \\
Newton-Raphson\\
$x_{n+1} = x-\frac{x-f(x)}{f\prime(x)}$\\
Identidad de Euler\\
$F = e - v+ 2$\\
Dijkstra\\
\begin{lstlisting}[language=C++, numbers=left,tabsize=4]
for i..n
		min=0
		for j..n
			if not vis[j] and dist[j]<dist[min]
				min=j
		vis[min]=true
		for j..n
			if G[min][j] and dist[j]>dist[min]+W[min][j]
				dist[j]=dist[min]+W[min][j]
\end{lstlisting}
Floyd-Warshall\\
\begin{lstlisting}[language=C++,numbers=left,tabsize=4]
for k..n
    for i..n
        for j..n
        	if G[i][k] and G[k][j] 
        		G[i][j] = Minumun(G[i][j],G[i][k]+G[k][j])
\end{lstlisting}
Bellman-Ford\\
\begin{lstlisting}[language=C++,numbers=left,tabsize=4]
//Usar arreglo auxiliar en caso de aristas negativas
min = INFTY
for i..V
	for j..E
		arista e=E[j]
		nuevo=min[e.a]+e.w
		if nuevo<min[e.b]
				min[e.b]=nuevo		
\end{lstlisting}


Kruskal\\
\begin{lstlisting}[language=C++,numbers=left,tabsize=4]
for i..E
	arista e=next()
	if pertenencia[e.a]!=pertenencia[e.b]
		int c=pertenencia[e.b];
		for j..n
			if pertenencia[j]==c
				pertenencia[j]=pertenencia[e.a];

\end{lstlisting}
\newpage
Prim\\
\begin{lstlisting}[numbers=left,language=C++,tabsize=4]
	dist[0]=0;
	vis[0]=false;
	for i=1..n
		dist[i]=INFTY
		vis[i]=false

	for i..n
		v=-1;
		for j..n
			if not vis[j] and (v==-1 or dist[j]<dist[v])
				v=j
		ans+=dist[v]
		vis[v]=true
		for j.. G[v].size()
			dist[G[v][j]]=minimun(dist[G[v][j]],W[v][j]);
\end{lstlisting}

Ford-Fulkerson

\begin{lstlisting}[numbers=left,language=C++,tabsize=4]
int camino(int a, int b)
	memset(vis,false, sizeof(vis))
	memset(padre,0,sizeof(padre))
	memset(Min,0,sizeof(Min))
	queue<int> q
	q.push(a)
	vis[a]=true
	Min[a]=INFTY
	while not q.empty()
		v = q.front()
		q.pop()
		for int i..G[v].size()
			if not vis[G[v][i]] and W[v][G[v][i]]!=0
				vis[G[v][i]]=true
				padre[G[v][i]]=v
				Min[G[v][i]]=min(Min[v],W[v][G[v][i]])
				q.push(G[v][i])
	if(vis[b])
		int i=b
		int t=Min[b]
		while(i)
			W[padre[i]][i]+=t
			W[i][padre[i]]-=t
			i=padre[i]
		return t
	return 0

int maximunFlow(int a, int b)
	ans=0
	while t=camino(a,b)
		ans+=t
	return ans
\end{lstlisting}

RMQ

\begin{lstlisting}[language=C++,numbers=left,tabsize=4]
class intervalo
	int a, b;
	bool contiene(int x)
		return x<=b and x>=a
	
	bool contiene(intervalo x)
		return a<=x.a and b>=x.b
	
	bool intersecta(intervalo x)
		return contiene(x.a) || contiene(x.b) || x.contiene(a) || x.contiene(b)
				
class nodo
	int suma
	nodo* izq
	nodo* der
	intervalo rango


class arbol
	nodo * raiz
	void crear(int arr[], int ini, int fin, nodo * aux)
		if(ini<fin)
			aux->rango=intervalo(ini,fin)
			int med=(ini+fin)/2
			aux->izq=new nodo()
			aux->der=new nodo()
			crear(arr, ini, med,aux->izq)
			crear(arr, med+1,fin, aux->der)
			aux->suma=aux->izq->suma+aux->der->suma
		else if(ini==fin)
			aux->mayor=arr[ini]
			aux->menor=arr[ini]
			aux->suma=arr[ini]
			aux->rango=intervalo(ini,ini)
			
	int suma(intervalo x,nodo*aux=NULL)
		if(aux==NULL)
			aux=raiz
		if(x.contiene(aux->rango))
			return aux->suma
		if(!x.intersecta(aux->rango))
			return 0
		int r=0
		if(aux->izq)
			r+=suma(x, aux->izq)
		if(aux->der)
			r+=suma(x, aux->der)
		return r
	
	void actualizar(int pos, int n, nodo * aux){
		if(aux->rango.contiene(pos))
			if(aux->rango.tam()==1)
				aux->suma=n
			else
				actualizar(pos, n, aux->izq)
				actualizar(pos, n, aux->der)
				aux->suma=aux->izq->suma+aux->der->suma
	
\end{lstlisting}

Knuth-Morris-Pratt

\begin{lstlisting}[language=C++, numbers=left, tabsize=4]
//Los indices inician en 1
pi[1]=0
for i=1..cadPat.size()
    c=cadPat[i+1]
    v=pi[i]
    while cadPat[v+1]!=c and v!=0
        v=pi[v]
    if cadPat[v+1]==c
        pi[i+1]=v+1
    else
        pi[i+1]=0

while cadTex.size()-k+1 >= cadPat.size()
    while j <= cadPat.size() and cadTex[i]==cadPat[j]
        i++
        j++
    if j>cadPat.size()
        res++ //Match
    if pi[j-1]>0
        k=i-pi[j-1]
    else
        if i==k
            i++
        k=i
    }
    if j>1
        j=pi[j-1]+1;
\end{lstlisting}

\newpage
Teor\'ia de Juegos\\
\begin{itemize}
	\item Ambos juegan de manera \'optima, conocen el juego
	\item Buscar una soluci\'on mediante DP y buscar un patr\'on
	\item Problema de las botellas:
		\begin{itemize}
			\item $n$  botellas en una mesa
			\item $k$  Cada turno, un jugador toma de $1$ a $k$ botellas.
			\item El jugador que toma la ultima botella es el perdedor
			\item Un estado $x$ es ganador si existe un caso perdedor en el intervalo $n-k,k$
			\item En caso contrario el estado es perdedor
			\item Ejemplo: $n=15,k=3
				\begin{array}{|c|c|c|c|c|c|c|c|c|c|c|c|c|c|c|c|c}
					\hline
					n&0&1&2&3&4&5&6&7&8&9&10&11&12&13&14&15\\
					\hline
					e&G&P&G&G&G&P&G&G&G&P& G& G& G& P& G&G\\
					\hline
				\end{array}
				$
			\item Buscar patrones, $e$ es P si $ 1 \equiv e (\mod k+1) $
			\end{itemize}
			Minimax
			\begin{itemize}
			\item Cada movimiento aumenta en $x$ la puntuacion de un jugador y la reduce en $-x$ para el oponente
			\item Utilizar recursi\'on memorizada. (De preferencia, usar MAP)
			\end{itemize}
			Juego del NIM
			\begin{itemize}
			\item Existen varias torres de ladrillos. Dos jugadores juegan alternadamente y toman cualquier cantidad de ladrillos, pero solo de una torre.
			\item El que tome el ultimo ladrillo gana.
			\item Cada torre tiene $t_1, t_2, t_3, ... t_n$ ladrillos
			\item El primer jugador gana si y solo si la suma nim $t_1 \oplus t_2 \oplus t_3 \oplus ... \oplus t_n$ es distinta de 0
			\item Para multiples juegos simultaneos, tambien aplica la regla del NIM
		\end{itemize}
\end{itemize}
\end{document}